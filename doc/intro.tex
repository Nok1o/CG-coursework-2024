\ssr{ВВЕДЕНИЕ}

Задача построения реалистичных трехмерных сцен, учитывающих оптические свойства поверхностей и источников света, является центральной в компьютерной графике. Для решения этой задачи было предложено множество алгоритмов. Генри Гуро и Том Фонг предложили алгоритмы реалистичной раскраски поверхностей, Эд Кэтмул ввел концепцию z-буфера, а Тернер Уиттед предложил трассировку лучей.

Цель работы –- реализовать алгоритм построения реалистичных трехмерных сцен объектов.

Для достижения поставленной цели необходимо решить следующие задачи:
\begin{enumerate}
	\item Проанализировать предметную область, рассмотреть известные подходы и алгоритмы;
	\item Спроектировать ПО для построения реалистичных сцен;
	\item Реализовать выбранные алгоритмы для построения трехмерной сцены;
	\item Исследовать характеристики разработанного ПО.
\end{enumerate}

\clearpage
