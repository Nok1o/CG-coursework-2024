\addcontentsline{toc}{chapter}{СПИСОК ИСПОЛЬЗОВАННЫХ ИСТОЧНИКОВ}
\begin{thebibliography}{}
	\bibitem{lit1} Простая модель освещения [Электронный ресурс]. URL: https://ychebnikkompgrafblog.wordpress.com/5-2-простая-модель-освещения/ (дата обращения 08.10.24);
	\bibitem{lit2} Алгоритм Робертса удаления невидимых граней [Электронный ресурс]. URL: https://cgraph.ru/node/495 (дата обращения 06.10.24);
	\bibitem{lit3} Алгоритм удаления невидимых граней, использующий z-буфер [Электронный ресурс]. URL: https://ychebnikkompgrafblog.wordpress.com/4-6-алгоритм-использующий-z-буфер/ (дата обращения 06.10.24);
	\bibitem{lit4} Метод обратной трассировки лучей [Электронный ресурс]. URL: https://cyberleninka.ru/article/n/metod-pryamoy-i-obratnoy-trassirovki/viewer (дата обращения 06.10.24);
	\bibitem{lit5} Закрашивание. Рендеринг полигональных моделей [Электронный ресурс]. URL: https://intuit.ru/studies/professional\_skill\_improvements/1283/courses/70/lecture/21/08?page=2 (дата обращения 06.10.24);
	\bibitem{lit6} Построение теней [Электронный ресурс]. URL: https://stratum.ac.ru/education/textbooks/kgrafic/additional/addit28.html (дата обращения 06.10.24);
	\bibitem{lit7} An Introduction to Ray Tracing [Электронный ресурс] 
	\bibitem{lit8} Fast Minimum Storage Ray/Triangle Intersection [Электронный ресурс]. URL: https://web.archive.org/web/20170517125238/http://www.cs.virginia.edu/~gfx/Courses/2003/ImageSynthesis/papers/Acceleration/Fast\%20MinimumStorage\%20RayTriangle\%20Intersection.pdf
\end{thebibliography}
