\ssr{ПРИЛОЖЕНИЕ А}
\begin{center}
	\begin{lstlisting}[label={lst:CubeIntersection}, caption={Алгоритм поиска точки пересечения луча с кубом (начало)}]
		public override bool IntersectRay(Ray ray, out double t, out Vector3 normal)
		{
			t = double.MaxValue;
			normal = new Vector3(0, 0, 0);
			double tMin = (Min.X - ray.origin.X) / ray.dir.X;
			double tMax = (Max.X - ray.origin.X) / ray.dir.X;
			
			if (tMin > tMax) (tMin, tMax) = (tMax, tMin);
			
			double tyMin = (Min.Y - ray.origin.Y) / ray.dir.Y;
			double tyMax = (Max.Y - ray.origin.Y) / ray.dir.Y;
			
			if (tyMin > tyMax) (tyMin, tyMax) = (tyMax, tyMin);
			if ((tMin > tyMax) || (tyMin > tMax)) return false;
			
			if (tyMin > tMin) tMin = tyMin;
			if (tyMax < tMax) tMax = tyMax;
			
			double tzMin = (Min.Z - ray.origin.Z) / ray.dir.Z;
			double tzMax = (Max.Z - ray.origin.Z) / ray.dir.Z;
			
			if (tzMin > tzMax) (tzMin, tzMax) = (tzMax, tzMin);
			
			double epsilon = 1e-8;
			
			if (tMin > tzMax + epsilon || tzMin > tMax + epsilon)
			return false;
			
			tMin = Math.Max(tMin, tzMin);
			tMax = Math.Min(tMax, tzMax);
		\end{lstlisting}
	\end{center}	
	
	\setcounter{lstlisting}{0}
	\clearpage
	\begin{center}
		\begin{lstlisting}[label={lst:CubeIntersection}, caption={Алгоритм поиска точки пересечения луча с кубом (конец)}]
			if (tMin < 0)
			{
				t = tMax;
				if (tMax < 0) 
				return false;
			}
			else 
			t = tMin;
			
			Vector3 hitPoint = ray.origin + ray.dir * t;
			if (Math.Abs(hitPoint.X - Min.X) < epsilon) normal = new Vector3(-1, 0, 0);
			
			else if (Math.Abs(hitPoint.X - Max.X) < epsilon) normal = new Vector3(1, 0, 0);
			else if (Math.Abs(hitPoint.Y - Min.Y) < epsilon) normal = new Vector3(0, -1, 0);
			else if (Math.Abs(hitPoint.Y - Max.Y) < epsilon) normal = new Vector3(0, 1, 0);
			else if (Math.Abs(hitPoint.Z - Min.Z) < epsilon) normal = new Vector3(0, 0, -1);
			else if (Math.Abs(hitPoint.Z - Max.Z) < epsilon) normal = new Vector3(0, 0, 1);
			return true;
		}
	\end{lstlisting}
\end{center}

\clearpage

\ssr{ПРИЛОЖЕНИЕ Б}
Презентация к курсовой работе содержит 19 слайдов.