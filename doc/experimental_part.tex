\chapter{Исследовательская часть}

В данном разделе приведены технические характеристики устройства, на котором проводились замеры времени выполнения программного обеспечения, а также результаты исследования.

\section{Технические характеристики}
Исследование по замерам времени выполнения программы проводились на персональном компьютере, оснащенном следующими компонентами:
\begin{itemize}
	\item процессор Intel Core i7-12700KF 3.60~Ггц с 12 ядрами: 8 производительных и 4 энергоэффектиных;
	\item оперативной памятью объемом 32~Гб;
	\item операционная система Windows 11 Home 64-рязрядная.
	\end{itemize}
	
\section{Описание исследования}
В рамках данного исследования проводятся замеры времени растеризации сцены в зависимости от числа лучей, используемых для симуляции эффекта глубины поля. В исследовании участвуют две сцены программы: сцена с 6 стенами и двумя сферами, каждая из которых имеет коэффициент отражения 0.5, а также сцена с шестью стенами и конем: сфера при этом полностью зеркальна.

Для каждой сцены и каждого числа лучей замеры времени усредняются по 20 замерам.

\section{Результаты исследования}
Результаты исследования зависимости времени растеризации сцены объектов в зависимости от числа лучей, используемых для симуляции эффекта глубины поля, приведены в таблице~\ref{}.

Результаты визуализации полученных данных приведены на рисунках~\ref{}~-~\ref{}.






\section{Вывод}

\clearpage
