\ssr{ЗАКЛЮЧЕНИЕ}

В результате выполнения курсового проекта была достигнута его цель: было разработано программное обеспечение для визуализации сцен трехмерных объектов с изменяемыми параметрами. 

В ходе работы были решены все поставленные задачи:
\begin{itemize}
	\item задача была формализована в виде IDEF0 диаграммы;
	\item рассмотрены основные алгоритмы удаления невидимых линий и поверхностей, методы закраски, методы построения теней. На основании их анализа были выбраны алгоритмы и методы, наиболее подходящие для решения поставленной задачи: в качестве алгоритма удаления невидимых линий и поверхностей был выбран алгоритм трассировки лучей в силу своей универсальности и реалистичности результата, в качестве метода закраски был выбран метод Фонга, для построения теней был выбран метод теста теневым лучом;
	\item была проведена декомпозиция задачи с помощью диаграмм функционального моделирования IDEF0, были разработаны схемы реализуемых алгоритмов визуализации сцены объектов методом трассировки лучей;
	\item спроектированное ПО было реализовано с помощью выбранных языка программирования C\# и среды разработки Microsoft Visual Studio;
	\item было проведено исследование временной эффективности реализованного ПО.
\end{itemize}

В результате исследования было выявлено оптимальное число потоков для многопоточной обработки, а также был установлен рост времени визуализации трехмерной сцены при увеличении числа объектов на ней.